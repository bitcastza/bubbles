\documentclass[12pt,a4paper]{article}
\usepackage[utf8]{inputenc}

\begin{document}
\title{Current Practice at UCT Underwater Club}
\author{Kyle Robbertze}
\date{June 2018}
\maketitle

The UCT Underwater Club has equipment relating to self contained breathing
apparatus (SCUBA) diving and apnoea (free diving). This equipment is used for
courses and events and it is rented out to members each week during the
semester. All equipment is numbered and this is recorded whenever equipment is
rented out or used. Some equipment requires a yearly service, such as buoyancy
control devices, cylinders and regulators, while all equipment may need repairs
from time to time. Once equipment gets too old (twenty years for cylinders,
condition depending for everything else), it is condemned and removed from
service.

When equipment is rented out, the member lists the equipment and sizes required.
The gear reps then collect equipment from the gear room and pack it into a bag.
As each item is placed into the bag, the equipment's number is recorded on the
rental sheet (including the bag's number). Weights are recorded by total weight
required, as weights do not have numbers. The weight belt is also not recorded,
as this is taken as a given if weights are provided. The weights and weight
belts count as a single item for payment purposes. Once the gear is collected,
the member fills out their name, student number and cell number. They also have
the chance to check the equipment. Once they are happy that the equipment is in
working condition, they sign the rental sheet to indicate that they have checked
their equipment and that they accept full responsibility for the equipment for
the duration of the rental period. The gear rep checks the member's
certification and signs that he has done so. The gear rep then totals the amount
owed by the member and collects this. The total is calculated as a fixed amount
per item of equipment plus a deposit that will be returned to the member upon
return of the gear clean and in full.

Equipment is rented by members for a fixed amount of time and afterwards is
returned by the member. The gear rep checks that they numbers returned match
what was rented out and that the equipment was cleaned. If this is the case, the
deposit is returned. Any issues with gear are noted.

Committee members are allowed to rent equipment at no charge. They must just record
what equipment they took and when it was taken and returned.

Cylinders have a set date that they must be serviced each year. This date is a
year after its last visual inspection. Cylinders are made out of aluminium or
steel and are hydro-statically tested once every two years. They are visually
inspected once a year and certain aluminium cylinders (most of those
manufactured before 1995) must also have an eddy current test once a year.
Certain details about a cylinder must be recorded to comply with South African
legislation. These details include the last hydro and visual test date, the
manufacturer, the date of manufacture, serial number, the volume it contains and
the working and test pressures the cylinder is rated to.

Cylinders are filled at the club. The filler records the start and end
pressures, last visual test date, filler, date and cylinder number or serial
number of each cylinder he or she fills. The compressor used to fill cylinders
has four storage banks attached to it. These are large cylinders, similar to the
cylinders that the compressor fills. Each of these banks must be visually and
hydro-statically tested once every ten years. The compressor also has a filter
that must be replaced every X number of running hours and a intake filter that
must be replaced once for every four filter replacements. The compressor must
also have a air purity test done every three months and the results logged.
\end{document}
